% !TEX root = ../my-thesis.tex
%
\chapter{Einleitung}
\label{sec:intro}
\cleanchapterquote{That’s the thing about people who think they hate computers.  What they really hate is lousy programmers.}{Larry Niven}{(Amerikanischer Science-Fiction Autor)}

In diesem Kapitel wird zunächst die Motivation, welche hinter dieser Arbeit steht, erläutert, dann die Zielsetzung definiert und zuletzt ein Ausblick über Inhalt und Struktur gegeben.

\label{sec:intro:motivation}
\section{Motivation}

Im Rahmen des Studiums gibt es viel Wissenswertes rund um die Universität, das für Studierende - insbesondere für Studienanfänger - relevant oder zumindest interessant ist. Dazu zählen allgemeine Informationen wie Nachrichten über die Universität und deren Fachbereiche und Institute, Veranstaltungen, Gebäude, Personen etc., aber auch wichtige Daten wie z.B. Informationen zu Vorlesungen, Anmeldefristen oder Prüfungsterminen. All diese Informationen sind aktuell auf verschiedene Webseiten verteilt, sodass man leicht den Überblick verlieren kann, wo welche Informationen zu finden sind und welche überhaupt erhältlich sind.

So gibt es eine Seite der Universität, auf der über allgemeine Dinge wie beispielsweise Veranstaltungstermine, Nachrichten aus diversen Forschungsabteilungen oder Details für Studieninteressierte informiert wird. Dann verfügen die meisten Institute ebenfalls über eine eigene Webseite, auf der unter anderem Kontaktdaten der DozentInnen und MitarbeiterInnen, aber auch Details zu den angebotenen Lehrveranstaltungen gefunden werden können. Weiterhin gibt es die Webseite des Zentrums für Datenverarbeitung (ZDV), auf welcher die IT-Dienstleistungen der Universität dokumentiert sind, darunter etwa, wo es auf dem Campus PC-Pools oder Drucker gibt. Darüber hinaus gibt es dann noch Portale wie \textit{JOGU-StINe}, der \textit{JGU-Reader} oder Moodle, mit welchen die Anmeldung und Organisation von Vorlesungen, Seminaren, Übungen oder Prüfungen durchgeführt wird.

Um all diese Quellen sowie deren Informationen und Angebote zu bündeln, soll eine Webanwendung entwickelt werden, die es den Studierenden ermöglichen soll, einen Überblick über Interessantes, Wissenswertes und Wichtiges rund um die Universität und das Studium zu erhalten. Damit soll eine Erstanlaufstelle geschaffen werden, über welche man zukünftig (möglichst) alle Informationen rund um das Studium an der Johannes Gutenberg-Universität Mainz beziehen und selbiges organisieren können soll.

Gleichzeitig sollen neue Möglichkeiten im Bereich der Webentwicklungen untersucht werden, mit denen man eine solche Plattform realisieren kann. Das beinhaltet auch eine Analyse der damit lösbaren Probleme als auch der Vorteile gegenüber bisherigen Vorgehensweisen. Des weiteren soll eine Alternative zur derzeit entwickelten App ``Uni-Mainz`` geschaffen werden, damit später ein Vergleich der beiden Ansätze möglich ist.

\section{Zielsetzung}
\label{sec:intro:aim}

Ziel dieser Arbeit ist es, eine Webanwendung zu programmieren, welche Daten aus verschiedenen Quellen bündelt. Damit soll eine Anlaufstelle geschaffen werden, die übersichtlich darstellt, welche Informationen es gibt und - sofern sie nicht in der Anwendung selbst angeführt werden - wo diese zu finden sind. Diese Webanwendung soll als sogenannte \acf{PWA} implementiert werden. Als \acs{PWA} werden Webanwendungen bezeichnet, die nicht nur in den gängigen Internetbrowsern auf verschieden Plattformen aufgerufen werden können, sondern (sofern dies vom verwendeten Betriebssystem unterstützt wird) auch als eigenständige Anwendung installiert werden können. Dadurch lässt sich eine Progressie Web App bedienen wie eine native Anwendung. Weitere Eigenschaften von PWAs werden in Kapitel \ref{sec:technologies:pwa} aufgelistet und erläutert. Daraus ergibt sich, dass die Webanwendung auf Desktop - und Mobilgeräten (möglichst) gleichermaßen benutzbar sein soll. Diese Webanwendung wird mit dem Framework ``Angular`` implementiert. Warum gerade dieses Framework verwendet wird und welche Alternative es gäbe, wird in Kapitel \ref{sec:technologies:frameworks} diskutiert; eine ausführliche Beschreibung von Angular erfolgt in Kapitel \ref{sec:technologies:angular}.

\section{Aufbau der Arbeit}
\label{sec:intro:structure}

\textbf{Kapitel \ref{sec:technologies}} \\[0.2em]
Dieses Kapitel beinhaltet eine Übersicht über die verwendeten Technologien, die in Frage kommenden Frameworks/Libraries und die eingesetzten Programmiersprachen. Insbesondere wird hier das für dieses Projekt verwendete Framework Angular ausführlich beschrieben und diskutiert, welche Vorteile es gegenüber möglichen Alternativen bietet. Außerdem werden zusätzlich genutzte Pakete und Ressourcen angeführt und kurz beschrieben.

\textbf{Kapitel \ref{sec:programming}} \\[0.2em]
Die Programmierung der Webanwendung sowie die dafür zu treffenden Vorbereitungen werden in Kapitel \ref{sec:programming} wiedergegeben und geschildert. Dabei wird dort zunächst eine Übersicht über die Inhalte gegeben, die diese Webanwendung umfassen soll sowie besondere Hindernisse angeführt, die die Entwicklung etwas beeinflusst haben.

\textbf{Kapitel \ref{sec:conclusion}} \\[0.2em]
Abschließend fasst Kapitel \ref{sec:conclusion} die Arbeit zusammen und gibt einen Ausblick darüber, welche Ziele erreicht, welche Bestandteile eventuell weiter ausgebaut und welche Inhalte der Webanwendung noch hinzugefügt werden könnten.