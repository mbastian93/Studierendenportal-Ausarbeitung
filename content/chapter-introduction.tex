% !TEX root = ../my-thesis.tex
%
\chapter{Einleitung}
\label{sec:motivation}
\section{Motivation}

\cleanchapterquote{That’s the thing about people who think they hate computers.  What they really hate is lousy programmers.}{Larry Niven}{(American science fiction auhtor)}

Im Rahmen des Studiums gibt es viel Wissenswertes rund um die Universität, das für Studierende - insbesondere für Studienanfänger - relevant oder zumindest interessant ist. Dazu zählen allgemeine Informationen wie Nachrichten über die Universität und deren Fachbereiche und Institute, Veranstaltungen, Gebäude, Personen etc. aber auch wichtige Daten wie z.B. Informationen zu Vorlesungen, Anmeldefristen oder Prüfungstermine. All diese Informationen sind aktuell auf verschiedene Webseiten verteilt, sodass man leicht den Überblick verlieren kann, wo welche Informationen zu finden sind und welche überhaupt erhältlich sind.
Um all diese Quellen und deren Informationen zu bündeln, soll eine Webanwendung programmiert werden, die es den Studierenden erleichtern soll, einen Überblick über Interessantes, Wissenswertes und Wichtiges rund um die Universität und das Studium zu bekommen.

\section{Zielsetzung}
\label{sec:intro:aim}

Ziel dieser Arbeit ist es, eine Webanwendung zu Programmieren, die Daten aus verschiedenen Quellen bündelt. Damit soll eine Anlaufstelle geschaffen, die übersichtlich darstellt, welche Informationen es gibt und - sofern sie nicht in der Anwendung selbst dargestellt werden - wo diese zu finden sind. Welche Inhalte dies genau sein sollen, wird in Kapitel \ref{sec:contents} beschrieben.
Diese Webanwendung soll als sogenannte Progessive Web App (kurz: PWA) implementiert werden. Als PWA werden Webanwendungen bezeichnet, die nicht nur in den gängigen Internetbrowsern auf verschieden Plattformen aufgerufen werden können, sondern (sofern dies vom verwendeten Betriebssystem unterstützt wird) auch als eigenständige Anwendung installiert werden können. Dadurch lässt sich eine Progressie Web App bedienen wie eine native Anwendung. Weitere Eigenschaften von PWAs werden in Kapitel \ref{sec:technologies:pwa} aufgelistet und erläutert.
Diese Webanwendung wird mit dem Framework "Angular" implementiert. Warum gerade dieses Framework verwendet wird und welche Alternative es gäbe, wird in Kapitel \ref{sec:technologies:frameworks} diskutiert; Eine ausführliche Beschreibung von Angular erfolgt in Kapitel \ref{sec:technologies:angular}

\section{Aufbau der Arbeit}
\label{sec:intro:structure}

\textbf{Kapitel \ref{sec:contents}} \\[0.2em]

\textbf{Kapitel \ref{sec:technologies}} \\[0.2em]

\textbf{Kapitel \ref{sec:programming}} \\[0.2em]

\textbf{Kapitel \ref{sec:conclusion}} \\[0.2em]

