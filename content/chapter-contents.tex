% !TEX root = ../my-thesis.tex
%
\chapter{Inhalte der Webanwendung}
\label{sec:contents}

\cleanchapterquote{}{}{()}

\section{Nachrichten}
\label{sec:contents:news}
Im ersten Modul sollen Nachrichten und Neuigkeiten der Universität und des Zentrums für Datenverarbeitung (ZDV) angezeigt werden. Dabei handelt es sich bei den Quellen um sogenannte RSS-Feeds

\section{Campus-Karte}
\label{sec:contents:map}
Das zweite Modul umfasst eine Karte des Universitätscampus, auf der - nach Auswahl - verschiedene Gebäude der Universität angezeigt werden.

\section{Busfahrpläne}
\label{sec:contents:bus}
Zu den Bus- und Straßenbahnhaltestellen rund um die Universität werden im dritten Modul die Abfahrtszeiten der demnächst eintreffenden Busse und Straßenbahnen angeben.
Die Fahrpläne werden hier über die API des Rhein-Main Verkehrsverbundes bezogen.

\section{Personensuche}
\label{sec:contents:searchPerson}
Das vierte Modul bietet eine Suchmaske um nach Angestellten der Universität zu suchen, die im Informationssystem der Universität Mainz (UnivIS) hinterlegt sind.

\section{Veranstaltungen}
\label{sec:contents:events}
In diesem Modul werden die im UnivIS angegebenen Veranstaltung in chronologischer Reihenfolge aufgelistet.

\section{Speisepläne}
\label{sec:contents:canteens}
Nach Tag und Ausgabe sortierte Speisepläne zu den Mensen auf dem Universitätsgelände können in diesem Modul abgefragt werden.

\section{Wetter}
\label{sec:contents:weather}
Das siebte Modul zeigt die Wetterdaten der Messstation auf dem Campus an.

\section{Öffnungszeiten}
\label{sec:contents:officeHours}
Hier können Öffnungszeiten verschiedener Einrichtungen und Büros eingesehen werden.

\section{Authentifizierung}
\label{sec:contents:auth}
Dieses nicht sichtbare Modul umfasst einen Authentifizierungsprozess mittels OpenID Connect

\section{Bibliotheksausweis}
\label{sec:contents:bibID}
Nach erfolgreicher Authentifizierung wird in diesem Modul ein digital Bibliotheksausweis angezeigt.