% !TEX root = my-thesis.tex


% **************************************************
% Files' Character Encoding
% **************************************************
\PassOptionsToPackage{utf8}{inputenc}
\usepackage{inputenc}


% **************************************************
% Information and Commands for Reuse
% **************************************************
\newcommand{\thesisTitle}{Entwicklung eines Studierendenportals als Progressive Web App mit Angular}
\newcommand{\thesisName}{Marcel Bastian}
\newcommand{\thesisSubject}{Matrikelnr.: 2687696}
\newcommand{\thesisDate}{15.Juli 2018}
\newcommand{\thesisVersion}{My First Draft}

\newcommand{\thesisFirstReviewer}{Prof. Dr. André Brinkman}
\newcommand{\thesisFirstReviewerUniversity}{\protect{Johannes Gutenberg-Universität Mainz}}
\newcommand{\thesisFirstReviewerDepartment}{Zentrum für Datenverarbeitung}

\newcommand{\thesisSecondReviewer}{Dr. Hans-Jürgen Schröder}
\newcommand{\thesisSecondReviewerUniversity}{\protect{Johannes Gutenberg-Universität Mainz}}
\newcommand{\thesisSecondReviewerDepartment}{Institut für Informatik}

\newcommand{\thesisFirstSupervisor}{Prof. Dr. André Brinkman}
\newcommand{\thesisSecondSupervisor}{Dr. Hans-Jürgen Schröder}

\newcommand{\thesisUniversity}{\protect{Johannes Gutenberg-Universität Mainz}}
\newcommand{\thesisUniversityInstitute}{Institut für Informatik}
\newcommand{\thesisUniversityCity}{Mainz}
\newcommand{\thesisUniversityStreetAddress}{Staudingerweg 9b}
\newcommand{\thesisUniversityPostalCode}{55128}


% **************************************************
% Debug LaTeX Information
% **************************************************
%\listfiles


% **************************************************
% Load and Configure Packages
% **************************************************
\usepackage[english]{babel} % babel system, adjust the language of the content
\PassOptionsToPackage{% setup clean thesis style
    figuresep=colon,%
    sansserif=false,%
    hangfigurecaption=false,%
    hangsection=true,%
    hangsubsection=true,%
    colorize=full,%
    colortheme=jgured,%
    bibsys=bibtex,%
    bibfile=bib-refs,%
    bibstyle=numeric,%
    wrapfooter=false,%
}{cleanthesis}
\usepackage{cleanthesis}

\hypersetup{% setup the hyperref-package options
    pdftitle={\thesisTitle},    %   - title (PDF meta)
    pdfsubject={\thesisSubject},%   - subject (PDF meta)
    pdfauthor={\thesisName},    %   - author (PDF meta)
    plainpages=false,           %   -
    colorlinks=false,           %   - colorize links?
    linkcolor=ctcolormain,      %   - link color (e.g., TOC)
    citecolor=ctcolormain,      %   - cite color
    pdfborder={0 0 0},          %   -
    breaklinks=true,            %   - allow line break inside links
    bookmarksnumbered=true,     %
    bookmarksopen=true          %
}
